\documentclass[12pt,conference]{IEEEtran}
\usepackage{amsmath}
\usepackage[spanish]{babel} %Definir idioma español
\usepackage[sort&compress]{natbib}
\usepackage[utf8]{inputenc} %Codificacion utf-8


\begin{document}

\title{Informe}

\author{José David Mamani Vilca}% <-this % stops a space

\onecolumn

\maketitle

\section{FASTP / FASTN: fast sequence similarity searching}

FASTP fue un programa diseñado originalmente para realizar búsquedas de similitud en proteínas. Fue desarrollado en lenguaje C para una VAX 11/780 bajo el sistema operativo UNIX. Debido al crecimiento exponencial de la información genética y a la velocidad limitada de la memoria en los años 80', se desarrollaron una serie de heurísticas y técnicas que solo demandan cantidades moderadas de memoria además de una unidad de disco con la capacidad suficiente para almacenar la librería con la secuencia de proteínas.

Actualmente, tanto FASTP como FASTN forman parte del legado Formato FASTA, un formato de fichero informático muy popular en Bioinformática. \citep{lipman1985rapid}

\medskip

\section{National Center for Biotechnology Information (NCBI) created at NIH/NLM }


Centro Nacional para la Información Biotécnologica (NCBI) forma parte de la Biblioteca Nacional de Medicina(NLM) en Estados Unidos, una rama de los Institutos Nacionales de Salud (NIH). Fue fundado el 4 de Noviembre de 1988 y esta localizado en Bethesda, Maryland. Tiene por misión ser un referente en cuanto a Información Biomolecular. Almacena y actualiza información referente a secuencias genómicas en GenBank (Base de Datos con secuencuas genéticas), artículos referentes a las áreas  de Biomedicina, Biotécnología, Bioquímica, Genética y Genómica, enfermedades genéticas en humanos y cua	lquier dato resaltante en términos biotécnologicos.

\medskip

\section{EMBnet network for database distribution}

European Molecular Biology network (EMBnet) es una red internacional compuesta por científicos y grupos interesados en otorgar y mejorar los servicios bioinformáticos. Esta red  compuesta por 37  nodos y repartidos a lo largo de 32 países busca otorgar acceso a bancos de datos bioinformáticos, a software especializado y a suficentes recursos computacionales. Desde su fundación en 1988, EMBnet pasó de ser una red meramente informal  de individuos con acceso a bases de datos biológicas, a la única organización mundial en donde sus profesionales buscan expandir el campo de la genética y la biología molecular. \citep{d200920}


\medskip


\section{BLAST: fast sequence similarity searching}

Basic Local Alignment Search Tool es un algoritmo para comparar información de secuencias  biológicas primarias tales como las secuencias de aminoácidos en proteínas. Es uno los programas bioinformáticos más utilizados siendo la heurística que utiliza una de las más eficientes en cuestiones de tiempo y por ende resultando en uno de los algoritmos más prácticos al momento de analizar las enormes bases de datos con genoma disponible. BLAST es posterior a FASTA resultando ser más rápido al solo ubicar patrones significativos en las secuencias.  \citep{altschul1990basic}

\medskip

\section{EST: expressed sequence tag sequencing}

Expressed Sequence Tag es una pequeña subsecuencia de una secuencia nucleotídica transcrita. Se utiliza en la identificación de genes, en el descubrimiento de genes y para la determinación de secuencias. Actualemente y con las mejoras en la identifcación del los EST se dispone de aproximadamente de 52 millones de EST's en bases de datos públicas (Por ejemplo en GenBank). Los ESTs son herramientas poderosas al momento de hallar un gen determinado dentro del genoma dado que reducen en gran medida el tiempo de ubicación, siendo algunos ejemplos la ubicación de los genes que causan la enfermedad de Alzheimer o el cáncer de Cólon. \citep{adams1991complementary} \citep{wiki_1}

\medskip

\bibliographystyle{abbrv}
\bibliography{biblio} 

\end{document}