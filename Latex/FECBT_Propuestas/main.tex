\documentclass[12pt,conference]{IEEEtran}
\usepackage{amsmath}
\usepackage[spanish]{babel} %Definir idioma español
\usepackage[utf8]{inputenc} %Codificacion utf-8


\begin{document}

\title{Justificaciones para propuestas de negocio}

\author{José David Mamani Vilca}% <-this % stops a space

\onecolumn

\maketitle


\section{Propuesta 1: Motor de Sucesos para Videojuegos de Alta Interactividad}
Para el estándar actual de videojuegos podemos decir casi con completa seguridad que todos ellos tienden a estar bajo un único grupo que los empequeñece rídiculamente. Si bien existen tipos y tipos, una gran cantidad de videojuegos se orientan a la narración de una historia la cuál describen paso  a paso como si de un guión se tratára. A este tipo de videojuegos se les conoce como Videojuegos Lineales y suelen ocupar gran parte de la Industria \textit{Gamer} (Aunque eso no implica que sean los más  jugados). Si bien durante décadas esta fue la \textit{moda}, las preferencias han ido evolucionando, resultando en este tipo, una experiencia que con el tiempo suele volver cansada y repetitiva. 

Dado esto se propone los videojuegos de Narración Interactiva.

La narración interactiva es parte integral de todo juego que busque sumergir al jugador dándole la capacidad de influenciar en el desarrollo del mismo. Se diferencia de otros tipos de inmersiones por su capacidad de soportar cualquiera de las posibles salidas que el jugador produzca además de la modificaciones drásticas en la línea argumental principal del juego.

El más importante reto dentro de la narración interactiva es mantener la coherencia de los sucesos. Al permitirse un
grado de influencia bastante alto por parte del jugador, el desarrollo de sucesos debe coincidar con la línea argumental principal que maneja el juego y acorde a esta con los sucesos previos al actual. Todo con el popósito de evitar posibles inconsistencias previas a la conclusión de la historia. Alcanzar tal objetivo puede, sin embargo, ser bastante complejo de alcanzar. La coherencia de la historia se ve influenciada tanto de forma inconsciente como consciente por parte del jugador. Imaginemos el caso de un jugador incapaz de estar al tanto de la historia en un determinado tiempo. Sus acciones podrían hasta ser incongruentes con el contexto narrativo actual, y aún así, el sistema debería ser capaz de generar líneas argumentales lo suficientemente coherentes como para permitirle al jugado retomar la historia. O el caso de un jugador ciertamente conocedor de la historia, pero interesado en conocer el desenvolvimiento de la línea narrativa al tomarse decisiones totalmente opuestas a las sugeridas por el contexto principal. En este punto el sistema no solo debería conservar la capacidad de mantener la coherencia, sino además, de generar líneas argumentales que aproximen al jugador lo más posible al propósito del argumento principal.

El desarrollo y la instauración de un motor de sucesos que permitan desarrollar una historia interactivamente daría inicio a una nueva generación de videojuegos  que asemejen el contexto ficticio a la realidad más próxima. De hecho, una implementación dotada de suficientes líneas arguementales sería capaz de emular, en pequeñas proporciones, un pequeño mundo virtual capaz de proponer retos a sus jugadores de manera aleatoria y a la vez autosostenible.


\section{Propuesta 2: Sistema de Canalizado para Riego Automático}

Si bien no es un problema muy extendido en la actualidad (especialmente para las áreas rurales), los sistemas de riego actuales están bajo el potencial de convertirse en un serio problema para los años venideros en dónde el consumo de agua será mayor y por ende más estrictamente controlado.

Actualmente los agricultores en áreas rurales marcan turnos en los cuáles señalan los horarios que tomarán para el regadío de sus cultivos. Si bien es un método funcional y a la vez práctico; requiere de un alto grado de coordinación y a la vez eficiencia. Esto evidentemente derivará en problemas, especialmente en casos en donde varios intenten escoger el mismo turno.  Otro problema deriva en turnos que no hayan sido utilizados, o turnos excesivamente largos para plantas que no requieren de una gran cantidad de agua. En todos estos casos, el principal problema trasciende al desperdicioo de agua no utilizada.

La instauración de un sistema que le permita a los agricultores señalar los turno que tomarán (de forma remota si así se requiere), permitiría un mejor uso del agua y por ende un desperdicio mínimo de la misma. Cancelar turnos, extender aquellos que quedaron muy cortos o simplemente pedir un turno de manera súbita podrían realizarse de manera rápida y directa sin necesidad de esperar un permiso en caso sea necasario.s


\end{document}